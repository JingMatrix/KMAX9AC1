\documentclass{article}

% Packages
\usepackage[T1]{fontenc} % For Latin characters
\usepackage[utf8]{inputenc} % For UTF-8 encoding
\usepackage[french]{babel} % For French language support
\usepackage{titlesec}
\usepackage{lipsum}
\usepackage{hyperref} % For email functionality
\usepackage{graphicx}

% Title
\title{Projet Deep Learning - Big Data MAPI3 2024}
\author{
    L\'eo And\'eol \\
    \href{mailto:leo.andeol@math.univ-toulouse.fr}{leo.andeol@math.univ-toulouse.fr}
}
\date{24 Janvier 2024}

\begin{document}


\maketitle



Ce projet constitue la deuxième partie du projet BigData.

Vous devrez, pour l'évaluation, rendre un notebook python détaillé avec des commentaires justifiant les méthodes utilisées.

Vous devrez aussi présenter vos résultats et les méthodes en 20 minutes lors de la soutenance du 07/02

La performance de vos modèles ne sera pas utilisée comme critère d'évaluation. Nous nous intéresserons uniquement à la méthodologie utilisée, aux comparaisons et analyses produites, ainsi qu'à la variété d'expériences menées.
Vous pouvez faire la présentation et le notebook en anglais si vous préférez.


Vous êtes encouragés à:
\begin{itemize}
    \item utiliser les notebooks de TP comme base de travail.
    \item utiliser Google Colab pour l'entrainement de vos modèles.
    \item utiliser la bibliothèque keras pour l'entrainement (tensorflow et pytorch sont acceptés).
\end{itemize}

Vous devez rendre votre notebook avant le 06/02 à 23h59.
Vous devez nous le transmettre par mail à
\begin{itemize}
    \item \href{mailto:leo.andeol@math.univ-toulouse.fr}{leo.andeol@math.univ-toulouse.fr},
    \item \href{mailto:francois.malgouyres@math.univ-toulouse.fr}{francois.malgouyres@math.univ-toulouse.fr} et
    \item \href{mailto:francois.bachoc@math.univ-toulouse.fr}{francois.bachoc@math.univ-toulouse.fr}
\end{itemize}
en précisant dans l'objet du mail "Projet Deep Learning - Big Data MAPI3 2024".

\section{Sujet:}
Vous allez, dans ce projet, entraîner plusieurs réseaux de neurones à classifier des paysages. Vous utiliserez pour cela le dataset Landscapes disponible \href{https://github.com/ml5js/ml5-data-and-models/tree/master/datasets/images/landscapes}{ici}.

Il s'agit d'un dataset comprenant 4000 images appartenant à 7 catégories.

\begin{figure}
    \centering
    \includegraphics[width=.9\columnwidth]{fig.jpg}
    \caption{Exemples du dataset}
    \label{fig:fig}
\end{figure}


Vous devez :

\begin{enumerate}
    \item Séparer vos données en 2 : un jeu d'apprentissage $80 \%$ et un jeu de validation $20 \%$

    \item Entraîner un réseau de convolutions que vous aurez défini vous même

    \item Utiliser de manière individuelle différentes techniques permettant d'augmenter la capacité de généralisation afin d'améliorer votre modèle (data augmentation, dropout, etc...)

    \item Utiliser un réseau pré-entraîné pour améliorer vos performances

    \item Comparer les résultats produits par 2. 3. et 4. dans un tableau récapitulatif (utilisez le jeu de validation).

    \item Afficher quelques exemples sur lesquels votre réseau se trompe et afficher ses prédictions

    \item Visualiser vos projections dans l'espaces des features à l'aide d'une t-SNE sur un sous ensemble du dataset (quelques centaines d'images).

    \item A l'aide des features calculées par votre réseau, choisir une image au hasard dans le dataset et afficher les 3 images qui sont les plus proches au sens de la distance euclidienne. Vous pourrez, si vous le souhaitez, utiliser les plus proches voisins de Scikit-Learn

\end{enumerate}


\end{document}
