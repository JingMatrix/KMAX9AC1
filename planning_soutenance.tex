\documentclass[11pt,reqno]{amsart}

% pour passer au pdf
\usepackage{pslatex}

% pour l'usage du francais (compatible avec pdf)
%\usepackage[T1]{fontenc}
\usepackage[utf8]{inputenc}
\usepackage{times}
\usepackage[french]{babel}
\usepackage[a4paper]{geometry}

% pour les maths
\usepackage{amsmath}
\usepackage{amsfonts}
\usepackage{amsbsy}
\usepackage{amssymb}

% pour les graphiques
\usepackage{graphicx}
\usepackage{psfrag}
\usepackage{epsfig}

% pour les algorithmes
\usepackage{algorithm}
\usepackage{algorithmic}
\floatname{algorithm}{Algorithme}
\renewcommand{\algorithmicrequire}{\textbf{Entrée:}}
\renewcommand{\algorithmicensure}{\textbf{Sortie:}}
\renewcommand{\algorithmicwhile}{\textbf{Tant que}}
\renewcommand{\algorithmicdo}{\textbf{faire}}
\renewcommand{\algorithmicendwhile}{\textbf{Fin tant que}}

\def\RR{{\mathbb R}}
\def\NN{{\mathbb N}}

\newcommand {\PS}[2] {\langle #1 , ~#2 \rangle}
\newtheorem{projet}{Projet}
\DeclareMathOperator{\argmin}{Argmin}


%%%%%%%%%%%%%%%%%%%%%%%%%%%%%%%%%%%%%%%%%%
% A MODIFIER
\newcommand {\COURS} {Big Data}
\newcommand {\FORMATION} {Universit\'e Paul Sabatier \hfill Master MApI3, M2}
\def\ext{sci}
\def\lang{Scilab}
%%%%%%%%%%%%%%%%%%%%%%%%%%%%%%%%%%%%%%%%%%%


%\input{./liste_projets.tex}

\begin{document}
\addtolength{\baselineskip}{+0.1\baselineskip}
\pagestyle{empty}
{\sc \bf \noindent \FORMATION \\

}
\vspace{0.5cm}
\begin{center}
    {\bf \COURS \\
        \vspace{0.5cm}
        Soutenances des projets\\
    }
\end{center}
\vspace{1cm}

Durant {\bf la soutenance}, chaque groupe fera une présentation de son travail. Vous disposerez d'un vidéo-projecteur.

\vspace*{0.5cm}


Les soutenances seront le mercredi 7 février à partir de 10h00, en U6-202.

\begin{itemize}

    \item{10h00 -} Chiron, Colombel, Cousin, Debbaoui, Marion
    \item{10h30 -} Dallies, Lumet, Cart-Lamy
    \item{11h00 -} Turgut, Fourment, Dayan Teixido, Aharmouch
    \item{11h30 -} Ricci, Bezia, Touil, Fall

    %\item{12h40 -} : projet ?
\end{itemize}




\end{document}
